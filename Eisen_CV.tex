\documentclass[letterpaper,11pt]{article}

\usepackage{latexsym}
\usepackage{etaremune}
\usepackage[utf8]{inputenc}
\usepackage{setspace}
\usepackage[empty]{fullpage}
\usepackage{titlesec}
\usepackage{marvosym}
\usepackage[usenames,dvipsnames]{color}
\usepackage{verbatim}
\usepackage{enumitem}
\usepackage[hidelinks]{hyperref}
\usepackage{fancyhdr}
\usepackage[english]{babel}
\usepackage{tabularx}
\usepackage{fontawesome5}
\usepackage{multicol}
\setlength{\multicolsep}{-3.0pt}
\setlength{\columnsep}{-1pt}
\input{glyphtounicode}
\usepackage[official]{eurosym}

%----------FONT OPTIONS----------
% sans-serif
% \usepackage[sfdefault]{FiraSans}
% \usepackage[sfdefault]{roboto}
% \usepackage[sfdefault]{noto-sans}
% \usepackage[default]{sourcesanspro}

% serif
% \usepackage{CormorantGaramond}
% \usepackage{charter}


\pagestyle{fancy}
\fancyhf{} % clear all header and footer fields
\fancyfoot{}

\renewcommand{\headrulewidth}{0pt}
\renewcommand{\footrulewidth}{0pt}

% Adjust margins
\addtolength{\oddsidemargin}{-0.6in}
\addtolength{\evensidemargin}{-0.5in}
\addtolength{\textwidth}{1.19in}
\addtolength{\topmargin}{-.5in}
\addtolength{\textheight}{1.4in}


\urlstyle{same}

\raggedbottom
\raggedright
\setlength{\tabcolsep}{0in}

% Sections formatting
\titleformat{\section}{
  \vspace{-4pt}\scshape\raggedright\large\bfseries
}{}{0em}{}[\color{black}\titlerule \vspace{-5pt}]

% Ensure that generate pdf is machine readable/ATS parsable
\pdfgentounicode=1


\lhead{Katherine (Kate) Eisen--CV \thepage}

\begin{document}
\thispagestyle{empty}

\begin{center}
    {\Huge \scshape Katherine Eisen} \\ \vspace{1pt}
    Naturvetarvägen 6A, 223 62 Lund, Sweden \\ \vspace{1pt}
    \small \raisebox{-0.1\height}\faPhone\ +1 781-697-7825 ~ \href{mailto:katherine.eisen@biol.lu.se}{\raisebox{-0.2\height}\faEnvelope\  \underline{katherine.eisen@biol.lu.se}} ~ 
    \href{https://kate-eisen.github.io/}{\raisebox{-0.2\height}\faGlobe\ \underline{kate-eisen.github.io/}}  ~
    \href{http://github.com/kate-eisen}{\raisebox{-0.2\height}\faGithub\ \underline{github.com/kate-eisen}}
    \vspace{-8pt}
\end{center}

\section{Education}
\begin{tabular*}{1.0\textwidth}[t]{l@{\extracolsep{\fill}}r}
\textbf{PhD in Ecology and Evolutionary Biology} & {\textbf{2020}}\\
\end{tabular*}
Cornell University, Ithaca, NY, USA\\
Advisor: Dr. M. A. Geber\\
Committee Members: Dr. A. A. Agrawal, Dr. R. A. Raguso\\
Dissertation title: Species interactions affect the distribution and evolution of multiple floral traits in California native wildflowers\vspace{7pt}\\


\begin{tabular*}{1.0\textwidth}[t]{l@{\extracolsep{\fill}}r}
{\textbf{Bachelor of Arts in Environmental Studies}, \textit{magna cum laude} with Distinction} & {\textbf{2012}} \\
\end{tabular*}
Amherst College, Amherst, MA, USA \\
Advisor: Dr. J. S. Miller \\
Honors thesis: Forty-two years of forest measurements support the continuation of the Northeastern carbon sink \\


\section{Professional Experience}
\begin{tabular*}{1.0\textwidth}[t]{l@{\extracolsep{\fill}}r}
\textbf{NSF Postdoctoral Fellowship in Biology, Rules of Life track}& {\textbf{2021-2024}} \\
\end{tabular*}
Lund University, Lund,  Sweden\\
Supervisors: Drs. Magne Friberg and Anna Runemark \vspace{7pt}\\

\begin{tabular*}{1.0\textwidth}[t]{l@{\extracolsep{\fill}}r}
\textbf{Maternity Leave}& {\textbf{Spring 2021}} 
\end{tabular*}


\vspace{7pt}\begin{tabular*}{1.0\textwidth}[t]{l@{\extracolsep{\fill}}r}
\textbf{Research Associate}& {\textbf{2012-2014}}\\
\end{tabular*}
University of Guelph, Ontario, Canada\\
Caruso and Maherali Labs, Department of Integrative Biology\\



\section{Peer-Reviewed Publications}
\begin{etaremune}
\item\textbf{Eisen, K. E.}, Ma, R., and R. A. Raguso.  Among- and within-population variation in floral morphology and scent in a hawkmoth pollinated plant. Accepted (26 May 2022), American Jounal of Botany, contributed to a special issue on ``Approaches to the Study of Quantitative Fitness-Related Traits.'' 

\item \textbf{Eisen, K. E.}, Siegmund, G.-F., Watson, M. A., and M. A. Geber. \textbf{2021}. Variation in the location and timing of experimental severing demonstrates that the persistent rhizome serves multiple functions in a clonal forest understory herb. Functional Ecology doi: 10.1111/1365-2435.13970\\
\item \textbf{Eisen, K. E.}, Geber, M. A., and R. A. Raguso. \textbf{2021}. Emission rates of species-specific volatiles vary across communities of \textit{Clarkia} species: Evidence for multi-modal character displacement. The American Naturalist doi: 10.1086/715501\\
\item Godsoe, W., \textbf{Eisen, K. E.}, Sirianni, K., and D. Stanton. \textbf{2021}. Transmission bias's fundamental role in biodiversity change. Theoretical Ecology 14: 367-379. doi: 10.1007/s12080-020-00478-3 \\
\item \textbf{Eisen, K. E.}, Wruck, A.C., and M. A. Geber. \textbf{2020}. Floral density and co-occurring congeners alter patterns of selection in annual plant communities. Evolution 74: 1682-1698. \\
\item \textbf{Eisen, K. E.}, Campbell, D. R., Richards, E., and M. A. Geber. \textbf{2019}. Differences in flowering phenology are likely not the product of competition for pollination in \textit{Clarkia} communities. International Journal of Plant Sciences 180: 974-986. \textbf{Special issue on floral trait evolution}\\
\item Caruso, C. M., \textbf{Eisen, K. E.}, Martin, R. A., and N. Sletvold. \textbf{2019}. A meta analysis of the agents of selection on floral traits. Evolution 73: 4-14.\\
\item \textbf{Eisen, K. E.}, and M. A. Geber. \textbf{2018}. Ecological sorting and character displacement in communities of coexisting \textit{Clarkia} species. Journal of Evolutionary Biology 31: 1440–1458.\\

\newpage
\vspace*{2mm}
\item \textbf{Eisen, K. E.}, Case, A. L., and C. M. Caruso. \textbf{2017}. Variation in pollen presentation in \textit{Lobelia siphilitica}. International Journal of Plant Sciences 178: 79–84. \textbf{Featured Cover Article}\\

\item Caruso, C. M., \textbf{Eisen, K. E.}, and A. L. Case. \textbf{2016}. A phylogenetic analysis of gynodioecy and its correlates in the flowering plants. International Journal of Plant Sciences 177: 115–121. \textbf{Featured Cover Article}\\

\item \textbf{Eisen, K.} and A. Barker Plotkin. \textbf{2015}. Forty years of forest measurements support steadily increasing aboveground biomass in a maturing, \textit{Quercus}-dominant northeastern forest. Journal of the Torrey Botanical Society 142: 97-112.
\end{etaremune}


\section{Book Reviews and Digests}
\begin{etaremune}
\item \textbf{Eisen, K. E.} \textbf{2018}. Digest: A new approach to an old question: using experimental evolution to investigate the evolution of generalized pollination. Evolution 72: 2825–2827.\\
\item \textbf{Eisen, K. E.} \textbf{2017}. Digest: Trait variation in Mimulus provides new evidence for the joint action of ecological sorting and character displacement. Evolution 71: 1425–1427.\\
\item Petipas, R. H.* and \textbf{K. E. Eisen.}* \textbf{2016}. The Roots and Shoots of Plant Evolutionary Ecology. A review of G. P. Cheplick 2015. Approaches to Plant Evolutionary Ecology. Trends in Ecology \& Evolution 31: 409-410. *Authors contributed equally.\\
\end{etaremune}


\section{Teaching Experience}
\begin{tabular*}{1.0\textwidth}[t]{l@{\extracolsep{\fill}}r}
\textbf{Teaching Associate} & {\textbf{Summer 2020}}\\
Evolution for non-majors online (BioEE 1180), Cornell University \vspace{7pt}\\

\textbf{Module Developer \& Guest Instructor} & {\textbf{Fall 2019}}\\
Evolution for non-majors (BioEE 1180), Cornell University \vspace{7pt}\\  

\textbf{Guest Lecturer} & {\textbf{Spring 2019}}\\
Chemical Ecology (BioEE 3690), Cornell University\vspace{7pt}\\

\textbf{Teaching Assistant} & {\textbf{Spring 2019}}\\
Chemical Ecology (BioEE 3690), Cornell University\vspace{7pt}\\

\textbf{Teaching Assistant} & {\textbf{Fall 2015, Fall 2018}}\\
Introduction to Evolutionary Biology \& Diversity (BioEE 1780), Cornell University\\

\end{tabular*}

\section{Fellowships and Awards}
\begin{tabular*}{1.0\textwidth}[t]{l@{\extracolsep{\fill}}r}
\textbf{Postdoctoral Fellowship in Biology, Rules of Life track} & {\textbf{2021-2024}}\\
National Science Foundation\vspace{7pt}\\

\textbf{Outstanding Graduate Teaching Assistant} & {\textbf{Spring 2020}}\\
Cornell University Department of Ecology and Evolutionary Biology\vspace{7pt}\\

\textbf{Ed Ricketts Student Talk Award} & {\textbf{January 2018}}\\
American Society of Naturalists Stand-Alone Meeting, Pacific Grove, CA\vspace{7pt}\\

\textbf{Paul Fellowship for Study and Research in Absentia } & {\textbf{Spring 2016}}\\
Cornell University Department of Ecology and Evolutionary Biology; \vspace{7pt}\\

\textbf{Graduate Research Fellowship} & {\textbf{2015-2020}}\\
National Science Foundation\vspace{7pt}\\


\end{tabular*}


\begin{tabular*}{1.0\textwidth}[t]{l@{\extracolsep{\fill}}r}

\textbf{Lloyd I. Rosenblum Memorial Fund for Graduate Study	} & {\textbf{2014-2017}}\\
Amherst College\vspace{7pt}\\

\textbf{Presidential Life Sciences Fellowship} & {\textbf{2014-2015}}\\
Cornell University\\

\end{tabular*}



\section{External Grants}
\begin{tabular*}{1.0\textwidth}[t]{l@{\extracolsep{\fill}}r}

\textbf{Swedish Phytogeographical Society Research Scholarship} & {\textbf{2022}}\\
Swedish Phytogeographical Society\vspace{7pt}\\

\textbf{Erik Philip-Sörensens research grant} & {\textbf{2021}}\\
Erik Philip-Sörensens stiftelse\vspace{7pt}\\

\textbf{Fysiografen Research Grant} & {\textbf{2021}}\\
Royal Physiographic Society of Lund\vspace{7pt}\\

\textbf{Fysiografen Travel Grant} & {\textbf{2021}}\\
Royal Physiographic Society of Lund\vspace{7pt}\\

\textbf{43rd Symposium Travel Grant} & {\textbf{2019}}\\
New Phytologist Trust\vspace{7pt}\\

\textbf{Graduate Student Research Fellowships, First Prize} & {\textbf{2017}}\\
Torrey Botanical Society\vspace{7pt}\\

\textbf{Lewis and Clark Fund for Exploration and Field Research} & {\textbf{2017}}\\
American Philosophical Society\vspace{7pt}\\

\textbf{Rosemary Grant Award for Graduate Student Research} & {\textbf{2015}}\\
Society for the Study of Evolution\vspace{7pt}\\

\textbf{Graduate Student Research Award} & {\textbf{2015}}\\
Botanical Society of America\\
\end{tabular*}

\section{Cornell University Grants }
\begin{tabular*}{1.0\textwidth}[t]{l@{\extracolsep{\fill}}r}

{\textbf{Sustainable Biodiversity Fund} (Atkinson Center for a Sustainable Future)} & {\textbf{2017}}\vspace{7pt}\\

{\textbf{Research Travel Grant} (Graduate School)} & {\textbf{2017}}\vspace{7pt}\\

{\textbf{Betty Miller Francis ‘47 Fund} (E\&EB Department)} & {\textbf{2016}}\vspace{7pt}\\

{\textbf{Sigma Xi Grant in Aid of Research, Cornell Chapter Award}} & {\textbf{2015, 2018}}\vspace{7pt}\\

{\textbf{Conference Travel Grant}, (Graduate School)} & {\textbf{2015, 2017, 2018, 2019}}\vspace{7pt}\\

{\textbf{Paul P. Feeny and Richard B. Root Funds } (E\&EB Department)} & {\textbf{2015, 2016, 2017}}\vspace{7pt}\\

{\textbf{Andrew W. Mellon Student Research Grant } (CALS)} & {\textbf{2015, 2016, 2018}}\vspace{7pt}\\

{\textbf{Orenstein Fund} (E\&EB Department)} & {\textbf{2015}}\\

\end{tabular*}

\section{Invited \& Selected Presentations}

\begin{tabular*}{1.0\textwidth}[t]{l@{\extracolsep{\fill}}r}
\textbf{Department of Ecology \& Evolution}  & \textbf{October 2021}\\
\end{tabular*}
University of Chicago, Chicago, IL\\
Departmental Seminar. Drivers of variation in floral traits: from species interactions to molecular mechanisms. \vspace{7pt}\\

\begin{tabular*}{1.0\textwidth}[t]{l@{\extracolsep{\fill}}r}
\textbf{Biology Department}  & \textbf{October 2021}\\
\end{tabular*}
University of Turku, Turku, Finland\\
Departmental Seminar. Drivers of variation in floral traits: from species interactions to molecular mechanisms. \vspace{7pt}\\


\begin{tabular*}{1.0\textwidth}[t]{l@{\extracolsep{\fill}}r}
\textbf{Biodiversity, Earth \& Environmental Science Department}  & \textbf{March 2020}\\
\end{tabular*}
Drexel University, Philadelphia, PA\\
Invited oral presentation to Graduate Student Seminar Series. Ecological and evolutionary interactions between pollinator-sharing plants.\vspace{7pt}\\

\newpage
\vspace*{2mm}


\begin{tabular*}{1.0\textwidth}[t]{l@{\extracolsep{\fill}}r}
\textbf{Academy Research Seminar}  & \textbf{March 2020}\\
\end{tabular*}
Academy of Natural Sciences, Philadelphia, PA\\
Invited oral presentation. Character displacement across multiple phenotypic axes in \textit{Clarkia}. \vspace{7pt}\\





\begin{tabular*}{1.0\textwidth}[t]{l@{\extracolsep{\fill}}r}
\textbf{\boldmath${43^{rd}}$ New Phytologist Symposium}  & \textbf{July 2019}\\
\end{tabular*}
Zurich, Switzerland\\
Selected oral presentation \& poster presentation. Emission rates of species-specific volatiles change across communities of \textit{Clarkia} species: Evidence for character displacement in floral scent. \vspace{7pt}\\

\begin{tabular*}{1.0\textwidth}[t]{l@{\extracolsep{\fill}}r}
\textbf{II Joint Congress on Evolutionary Biology}  & \textbf{August 2018}\\
\end{tabular*}
Montpellier, France\\
Selected oral presentation. Does sharing pollinators affect floral trait evolution? Stories from the \textit{Clarkia} system. \\



\section{Contributed Presentation (since 2014)}

\begin{tabular*}{1.0\textwidth}[t]{l@{\extracolsep{\fill}}r}
{\textbf{Scandinavian Association of Pollination Ecologists (SCAPE)}, Warsaw, Poland }  & \textbf{October 2021}\\
\end{tabular*}
Oral presentation. The repeatability of floral scent across a geographic mosaic of coevolution in \textit{Oenothera cespitosa} ssp. \textit{marginata}. \vspace{7pt}\\

\begin{tabular*}{1.0\textwidth}[t]{l@{\extracolsep{\fill}}r}
{\textbf{American Society of Naturalists}, Pacific Grove, CA }  & \textbf{January 2020}\\
\end{tabular*}
Oral presentation. Are changes in flower size associated with changes in plant size and physiology? \vspace{7pt}\\

\begin{tabular*}{1.0\textwidth}[t]{l@{\extracolsep{\fill}}r}
{\textbf{Evolution}, Providence, RI}  & \textbf{June 2019}\\
\end{tabular*}
Oral presentation. Emission rates of species-specific volatiles change across communities of \textit{Clarkia} species: Evidence for character displacement in floral scent. \vspace{7pt}\\

\begin{tabular*}{1.0\textwidth}[t]{l@{\extracolsep{\fill}}r}
{\textbf{American Society of Naturalists}, Pacific Grove, CA }  & \textbf{January 2018}\\
\end{tabular*}
Oral presentation. Does sharing pollinators affect floral trait evolution? Stories from the \textit{Clarkia} system.\vspace{7pt}\\

\begin{tabular*}{1.0\textwidth}[t]{l@{\extracolsep{\fill}}r}
{\textbf{Ecological Society of America}, Portland, OR}  & \textbf{August 2017}\\
\end{tabular*}
Oral presentation. Ecological sorting and character displacement in communities of pollinator-sharing \textit{Clarkia} species.\vspace{7pt}\\

\begin{tabular*}{1.0\textwidth}[t]{l@{\extracolsep{\fill}}r}
{\textbf{Botany}, Edmonton, AB, Canada}  & \textbf{July 2015}\\
\end{tabular*}
Oral presentation. Sources of variation in pollen dispensing schedules of \textit{Lobelia siphilitica}. \vspace{7pt}\\

\begin{tabular*}{1.0\textwidth}[t]{l@{\extracolsep{\fill}}r}
{\textbf{Evolution}, Raleigh, NC}  & \textbf{June 2014}\\
\end{tabular*}
Oral presentation. A phylogenetic analysis of gynodioecy and its correlates in the flowering plants.\\

\section{Outreach \& Mentoring Activities}
\begin{tabular*}{1.0\textwidth}[t]{l@{\extracolsep{\fill}}r}
\textbf{Diversity Preview Weekend}  & \textbf{2017-2020}\\
\end{tabular*}
Cornell University\\
\begin{itemize}[noitemsep,topsep=0pt]
\item Coordinated participants’ housing, ran workshop on NSF GRFP, provided feedback on personal statements.\vspace{7pt}
\end{itemize}

\begin{tabular*}{1.0\textwidth}[t]{l@{\extracolsep{\fill}}r}
\textbf{Scientist Mentor Fellow}  & \textbf{2016-2018}\\
\end{tabular*}
Digging Deeper Project, PlantingScience\\
\begin{itemize}[noitemsep,topsep=0pt]
\item Attended pedagogical training session with secondary-school teachers; worked intensively with teachers on program implementation\vspace{7pt}\\
\end{itemize}

\begin{tabular*}{1.0\textwidth}[t]{l@{\extracolsep{\fill}}r}
\textbf{Undergraduate mentoring \& supervising}  & \textbf{2015-2020}\\
\end{tabular*}
Cornell University\\
\begin{itemize}[noitemsep,topsep=0pt]
\item Supervised and mentored two students conducting independent research projects and one work-study student.\vspace{7pt}\\
\end{itemize}

\newpage
\vspace*{2mm}

\begin{tabular*}{1.0\textwidth}[t]{l@{\extracolsep{\fill}}r}
\textbf{Master Plant Science Team Member}  & \textbf{2015-2016}\\
\end{tabular*}
PlantingScience\\
\begin{itemize}[noitemsep,topsep=0pt]
\item Served as a classroom liaison between teachers and scientist mentors\vspace{7pt}\\
\end{itemize}



\begin{tabular*}{1.0\textwidth}[t]{l@{\extracolsep{\fill}}r}
\textbf{Instructor}  & \textbf{2014-2015}\\
\end{tabular*}
Graduate Student School Outreach Program (GRASSHOPR), Cornell University\\
\begin{itemize}[noitemsep,topsep=0pt]
\item Designed and taught short course on biodiversity to a fourth-grade class\vspace{7pt}\\
\end{itemize}

\begin{tabular*}{1.0\textwidth}[t]{l@{\extracolsep{\fill}}r}
\textbf{Volunteer Educator}  & \textbf{2012-2014}\\
\end{tabular*}
Let's Talk Science, University of Guelph\\
\begin{itemize}[noitemsep,topsep=0pt]
\item Presented interactive activities on a range of STEM topics to classrooms and community groups\vspace{7pt}\\
\end{itemize}

\begin{tabular*}{1.0\textwidth}[t]{l@{\extracolsep{\fill}}r}
\textbf{Undergraduate mentoring \& supervising}  & \textbf{2012-2014}\\
\end{tabular*}
University of Guelph\\
\begin{itemize}[noitemsep,topsep=0pt]
\item Supervised eight work study students and mentored two honors thesis students.\\
\end{itemize}



\section{Professional Activities}
\textbf{Reviewer} (42 total manuscripts): Evolution (10); American Journal of Botany (5); Oecologia (5); Journal of Ecology (4); Annals of Botany (2); Biological Invasions (2); Functional Ecology (2); International Journal of Plant Sciences (2); New Phytologist (2); Evolutionary Ecology (1); Frontiers in Plant Science (1);  Journal of Plant Ecology (1); Journal of Plant Research (1); Nordic Journal of Botany (1); Plant Ecology \& Diversity (1); Proceedings B (1); Trees (1)\vspace{7pt}\\

\textbf{Graduate Student Representative}, faculty search in Organismal Biology, Cornell University (2019-2020)\vspace{7pt}\\

\textbf{Contributed Symposium Organizer}, ESEB 2022 Congress, Prague, Czech Republic\vspace{7pt}\\

\textbf{Meeting Organizer}, ${ECR^{2}}$ Online 2020; Cornell Ecology and Evolutionary Biology December Symposium (2015, 2016, 2017); Ontario Ecology, Ethology, and Evolution Colloquium (2014)\vspace{7pt}\\

\textbf{Vice Chair}, American Society of Naturalists Diversity Committee (2022)\vspace{7pt}\\

\textbf{Reviewer}, American Society of Naturalists Student Research Award (2022)\vspace{7pt}\\

\textbf{Member}, American Society of Naturalists Diversity Committee (2019-2021)\vspace{7pt}\\

\textbf{Member}, American Society of Naturalists Graduate Council (2019-2022)\vspace{7pt}\\

\textbf{Member}, Botanical Society of America; California Botanical Society; California Native Plants Society; Ecological Society of America; Sigma Xi; American Society of Naturalists; Society for the Study of Evolution; Torrey Botanical Society\\

\section{Special Skills \& Training}

\begin{tabular*}{1.0\textwidth}[t]{l@{\extracolsep{\fill}}r}
\textbf{Writing 7101: Writing in the Majors Seminar}  & \textbf{Spring 2019}\\
\end{tabular*}
Cornell University\\
\begin{itemize}[noitemsep,topsep=0pt]
\item Received training on teaching writing; discussed challenges associated with working with student writing.\vspace{7pt}\\
\end{itemize}

\begin{tabular*}{1.0\textwidth}[t]{l@{\extracolsep{\fill}}r}
\textbf{GET SET Certificate in “Understanding Undergraduate Learners”}  & \textbf{2018}\\
\end{tabular*}
Center for Teaching Innovation, Cornell University\\
\begin{itemize}[noitemsep,topsep=0pt]
\item Participated in a series of workshops on fostering inclusivity and diversity in the classroom, facilitating effective discussions, and how to support struggling students.\vspace{7pt}\\
\end{itemize}
\newpage
\vspace*{2mm}

\begin{tabular*}{1.0\textwidth}[t]{l@{\extracolsep{\fill}}r}
\textbf{Colman Leadership Program}  & \textbf{January 2018}\\
\end{tabular*}
Cornell University\vspace{7pt}\\



\begin{tabular*}{1.0\textwidth}[t]{l@{\extracolsep{\fill}}r}
\textbf{Leadership Development for Life Scientists (ALS 5100)}  & \textbf{Spring 2017}\\
\end{tabular*}
Cornell University\vspace{7pt}\\

\begin{tabular*}{1.0\textwidth}[t]{l@{\extracolsep{\fill}}r}
\textbf{Geometric Morphometrics in R}  & \textbf{Spring 2016}\\
\end{tabular*}
Transmitting Science, Barcelona, Spain\\
Instructor: Dr. Julien Claude, Assistant Professor, ISEM, France\vspace{7pt}\\



\begin{tabular*}{1.0\textwidth}[t]{l@{\extracolsep{\fill}}r}
\textbf{Scientific Communication Workshop (COMM 5660)}  & \textbf{Fall 2015}\\
\end{tabular*}
Cornell University

\end{document}